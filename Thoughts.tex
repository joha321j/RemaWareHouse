\documentclass[a4paper]{article}

\author{Johannes Ehlers Nyholm Thomsen}
% Danish symbols.
\usepackage[utf8]{inputenc}
\usepackage[danish]{babel}
\usepackage[T1]{fontenc}

% Clickable links
\usepackage{hyperref}


\begin{document}

\title{Udvikling af RemaWareHouse API}

\maketitle

\subsection*{Github link: \url{https://github.com/joha321j/RemaWareHouse}}

\subsection*{Enhed som klasse}
Jeg valgte at udvide data modellen fra det angivne ved at tilføje endnu en model,
der repræsenterer enheder.
Dette har jeg valgt at gøre, da jeg nemt kan forestille mig,
at de samme enheder vil gå igen: cm, kg, g, etc.

\subsection*{Generic Services}
For at undgå, at mine controllers bliver store gudeklasser har jeg valgt,
at de skal gøre brug af bagvedliggende services.

Først lavede jeg en service til hver af de simple klasser; 
enheder, leverrandører og kategorier.
Jeg opdagede dog, at alt for meget af koden gik igen i mellem hver service.
Derfor besluttede jeg mig for at benytte mig af generics,
så i stedet lavede jeg en generic service for hver CRUD operation.

Desværre kan min product klasse ikke bruge min generiske service til Read operationen,
da product klassen indeholder referencer til mine tre simple klasser.
Referencerne gør, at jeg er nød til at gøre brug af include statements,
så Entity Framework følger referencerne og læser alt dataen med ud.

Heldigvis er det ikke tilfældet for de tre andre CRUD operationer.

\subsection*{Create og Update product}
Når man forsøger at skabe eller opdatere et produkt,
checker API'en om de angivne kategori, enhed og leverrandør eksisterer.
Hvis mindst én af dem ikke findes returneres et BadRequest med statuskode 400.
Det har jeg valgt at gøre, da jeg ikke mener, 
at product endpointet skal kunne bruges til at lave nye kategorier, enheder
eller leverrandører.
De har alle hver deres endpoint. 

\subsection*{Data transfer objects}
Alle mine create og update endpoints gør brug af data transfer objects.
Dette har jeg valgt at gøre, så brugeren ikke har kontrol over objektets id - pånær når PUT endpoints bruger.

\subsection*{Valg af database}
Jeg har valgt at gøre brug af en SQLite database.
Jeg valgte at gøre det frem for en in-memory database, da jeg så slap for at skulle skrive kode, der seeder min database hver gang, men jeg i stedet kunne populere min database ved hjælp af mine ejne endpoints.
Heldigvis er det nemt at udskifte databasen - man skal blot ændre databasetypen i ens Startup klasse under ConfigureServices metoden.

\subsection*{Indkludering af underklasser i produkt}
Get endpointet i min ProductController tager 3 booleans:
\begin{enumerate}
    \item withUnit
    \item withCategory
    \item withSupplier
\end{enumerate}

Ideen med dette er, at man skulle kunne vælge, hvorvidt man ville have alt dataen som min Unit, Category og Supplier klasser indeholder, eller blot deres id'er.

\end{document}